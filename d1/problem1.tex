\section{1 Background Information}

The purpose of descriptive statistics is to quantitatively describe a collection of data by measures of central tendency, measures of frequency, and measures of variability. \\
\\Let x be a random variable that can take values from a finite data set x1, x2, x3, ..., xn, with each value having the same probability. \\
\\The minimum, m, is the smallest of the values in the given data set. (m need not be unique.)\\
\\The maximum, M, is the largest of the values in the given data set. (M need not be unique.)\\
\\The mode, o, is the value that appears most frequently in the given data set. (o need not be unique.)\\
\\The median, d, is the middle number if n is odd, and is the arithmetic mean of the two middle numbers if n is even.\\
%
% Arithmetic Mean
%
\\The arithmetic mean, $\mu$, is given by
\[ \mu = \frac{1}{n} \sum_{i=1}^{n} x_i \]\\
%
% Mean Absolute Deviation
%
\\The mean absolute deviation, MAD, is given by
\[ \text{MAD} = \frac{1}{n} \sum_{i=1}^{n} |x_i - \mu| \]\\
%
% Standard Deviation
%
The standard deviation, $\sigma$, is given by
\[ \sigma = \sqrt{\frac{1}{n}\sum_{i=1}^{n}(x_i - \mu)^2} \]\\
\\Let there be a system, called METRICSTICS (a portmanteau of METRICS and
STATISTICS), for finding m, M, o, d, $\mu$, MAD, and $\sigma$. The system must take as input a random number of data values and output its descriptive statistics. 


\vspace{100pt}
\section{2 Problem 1}
\vspace{10pt}
\subsection{2.1 Smart Goal}
\vspace{10pt}
\normalsize{Enhance the computational efficiency of the METRICSTICS calculator and optimize its usability across all functions through a series of iterative improvements to be completed by the end of the current semester, November 2023. }\\
\\The provided goal satisfies the SMART criteria:
\begin{itemize}
\item \textbf{Specific:} The goal is specific as it clearly outlines the objective to create a user friendly METRICSTICS with improved computational efficiency
\item \textbf{Measurable:} Specific metrics are defined to measure the computational efficiency of the goal such as user feedback,Task Success Rate,processing time in milliseconds etc. 
\item \textbf{Attainable:} The goal is both practical and attainable within the allotted timeframe and with the resources currently available. It does not entail the adoption of cutting-edge technologies or frameworks demanding extensive memory storage.
\item \textbf{Realistic:} The goal aligns with the overall purpose and objectives of METRICSTICS. It is realistically framed considering the needs of the users and how the enhancements will benefit them.
\item \textbf{Timely:} We have set a specific deadline to achieve the goal for completing the improvements which is the end of the current semester, November 2023
\end{itemize}

\subsection{2.2 QUESTIONS AND METRICS}
Below are the 12 QUESTIONS and metrics related to the defined GOAL  \\ \\
\textbf{Question 1:} How can we improve and measure the overall usability of the METRICSTICS system over iterations?
 \\
\textbf{Metrics:}
\begin{itemize}
    \item User satisfaction score calculated through LIKERT scale.
    \item User feedback and suggestions collected through surveys or interviews.
\end{itemize} 
\vspace{10pt}
\textbf{Question 2:} What is the average completion time for a new user to do a standard task in METRICSTICS? (Example - Computing the median (d)) ? \\
\textbf{Metrics:}
\begin{itemize}
    \item Usability of METRICSTICS using System Usability Scale (SUS) questionnaire.
    \item Percentage of users that completed stated activities successfully. 
\end{itemize}
\vspace{10pt}
\textbf{Question 3:} How do users rate the responsiveness and speed of METRICSTICS interface during various operations? \\
\textbf{Metrics:}
\begin{itemize}
    \item Measure the troublesome functionalities through User Surveys, execution time and error rate.
    \item Measure the Task Success Rate.
\end{itemize}
\vspace{10pt}
\textbf{Question 4:} What is the CPU utilization during the peak processing time for METRICSTICS when handling multiple datasets concurrently?
 \\
\textbf{Metrics:}
\begin{itemize}
    \item Measure the variation in processing time while executing a functionality across dataset of varying sizes.
    \item Measure the memory consumption by the application.
\end{itemize}
\vspace{10pt}
\textbf{Question 5:} What is the response time of METRICSTICS when generating descriptive statistics for datasets of varying sizes? \\
\textbf{Metrics:}
\begin{itemize}
    \item Measure the scalability for varying datasets using response time. 

    \item  Measure the resource efficiency by calculating response time per unit of resource consumption.
\end{itemize}
\vspace{10pt}
\textbf{Question 6:} Are there specific operations or calculations within METRICSTICS that tend to be more resource-intensive, and if so, which ones? 
 \\
\textbf{Metrics:}
\begin{itemize}
    \item Measure the execution time (in milliseconds) for resource-intensive operations and compare it to less intensive ones.
    \item Calculate the efficiency gain by comparing the resource consumption before and after optimization efforts for resource-intensive operations.
\end{itemize}
\vspace{10pt}
\textbf{Question 7:} Are there specific data formats or structures that result in higher memory usage when processed by METRICSTICS? \\
\textbf{Metrics:}
\begin{itemize}
    \item  Evaluate the memory consumption. 
\end{itemize}
\vspace{10pt}
\textbf{Question 8:}: What is the average number of clicks or interactions required for a user to locate and execute a specific calculation task in METRICSTICS? \\
\textbf{Metrics:}
\begin{itemize}
    \item Average clicks to find tasks.
    \item Time to locate and execute tasks.
\end{itemize}
\vspace{10pt}
\textbf{Question 9:} What strategies can be implemented to handle the even and odd sample size for a median? \\
\textbf{Metrics:}
\begin{itemize}
    \item Calculate the percentage of cases where the calculated median deviates from the expected median.
    \item Conduct user surveys to gauge user satisfaction.
\end{itemize}
\vspace{10pt}
\textbf{Question 10:} What is the average time taken by the development team to implement user interface enhancements or changes based on usability feedback? \\
\textbf{Metrics:}
\begin{itemize}
    \item Calculate the duration of the development cycle.
    \item Determine the ratio of usability feedback items resolved to the total development effort in one cycle.
\\
\end{itemize}
\vspace{10pt}
\textbf{Question 11:} How frequently do critical errors or system failures occur in METRICSTICS, and what measures can be taken to reduce these incidents?  \\
\textbf{Metrics:}
\begin{itemize}
    \item Conduct root cause analyses for critical errors and system failures to identify underlying issues.
    \item ATrack the trend in critical errors and system failures over time.
 \\
\end{itemize}
\vspace{10pt}
\textbf{Question 12:} How can we improve the team’s efficiency in maintaining and improving the usability of METRICSTICS over time? \\
\textbf{Metrics:}
\begin{itemize}
    \item Regular User-based survey for the improvement of usability.
    \item Reduce the delay between issue identification and development. \\
\vspace{10pt}
\vspace{100pt}
\end{itemize}


